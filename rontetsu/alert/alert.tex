\documentclass[11pt,a4paper,onecolumn,article]{jarticle}
\usepackage{otf}
\usepackage[top=30truemm,bottom=30truemm,left=25truemm,right=25truemm]{geometry}
\usepackage{url}
\usepackage[dvipdfmx]{hyperref}
\usepackage{pxjahyper}
\hypersetup{% hyperrefオプションリスト
setpagesize=false,
 bookmarksnumbered=true,%
 bookmarksopen=true,%
 colorlinks=true,%
 linkcolor=blue,
 citecolor=red,
}

\title{「『論理哲学論考』問題集(仮題)」に向けた原稿執筆に関する注意}
\date{}

\begin{document}
  \maketitle

  \section{入稿に関して}
  2016年5月1日に開催される「文学フリマ」に向けて、皆さんには原稿を執筆してもらいたく思います。その原稿レイアウトを\texttt{tex}ファイルとして配布するので、それを参考に執筆してください。

  入稿する会社は\href{http://www.inv.co.jp/~popls/}{「ポプルス」}を使います。今回は初めての出展なので少な目に刷ります。みなさんの手元に置いていく分と知り合いに配布する分も合わせて50部刷る予定です。

  料金ですが、大体2万円前後になる予定です。入稿の前に集金できるのがベストです。

  \section{締め切りについて}

  締め切りについてですが、納入してから一週間で刷り上がるので、搬入する時間も考えて4月の頭までには出来上がってほしいところです。上記の事情を鑑み、{\bf 2016年3月18日23:59}までとします。それまでに出版できるまで形を仕上げてください。コピー本ではなくオフセット本なので、締め切りを延ばしたりとかの対応は若干厳しいです。締め切りまでに納得のいく原稿を仕上げてください。
\end{document}
