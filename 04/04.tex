\documentclass[12pt,a4paper,onecolumn,openany,article]{jsarticle}
\usepackage{otf}
\usepackage{algorithm}
\usepackage[top=30truemm,bottom=30truemm,left=25truemm,right=25truemm]{geometry}

\title{TokyoTech Literature Club Reading Session Part4}
\author{Yusuke Izawa}
\date{December 24, 2015}

\begin{document}
\maketitle

\section{この読書会の進め方}
\begin{itemize}
  \item 
\end{itemize}

\section{前回の問いに対する応答}
\begin{itemize}
  \item 言語とは日常言語のことを言うのか?
  
  非日常的な言語を私たちは思考できない。そもそも、思考可能か不可能であるかの領域は言語によって引かれているはずであ り、事実から作られる命題は知覚可能なものであるから、それは日常言語で構成されているはずである。
  \begin{quote}
     4.002 | 日常言語から言語の論理を直接に読み取ることは人間には不可能である。
  \end{quote}  
   
  これは果たして日常言語からの脱却を目指していることを示唆しているのだろうか?ラッセルは「論考」が完全な理想言語を満たす条件を考察していると述べたが、これはラムジーによる指摘\footnote{野矢先生の本のp.134を参照}で誤りであることがわかっている。したがって、言語とは完全に日常言語と異なるものではなく、日常言語との格闘の末生み出されるもの、と解釈すべきだろう。
  
  \item 「分析可能」とは何を以ってそう言うのか?
  
  それは点灯空間の例を見れば明らかだろう。起こりうる状況を真/偽で過不足なく記述している。つまり「完全に記述できる」という状況を以って「分析可能」というのである。
  
  また、以降の内容で出てくるが、「分析可能」を「分解/構成」が可能であると捉えるのもいいかもしれない。
  
\end{itemize}

\section{前回までのおさらい}
本章は前章までと比べるとテクニカルな部分が多い。それは、真理操作というプロセスの説明が多く、基礎的な概念の説明は前章まででほぼ終えているからである。したがって、本レジュメも処理の解説に重きをおく。まずは、これまでの構図を簡単に振り返ろう。

まず、論考は現実世界と日常言語から出発した。日常言語を分析し、再び日常言語を構成するという往復運動を行い続け、思考可能性の全体を明確に見通すことを目指している。名や対象を分析するのではなく、理想言語を追求するのではない。あくまでも、始まりと終わりは日常言語にある。

\subsection{分析}

分解と構成の繰り返しによって分析は執り行われる。これを簡単に言うと「{\bf{バラして組み合わせる}}」ことである。バラし方にもルールがあって、それは論理形式に則っていなければならないということである。

\subsubsection{分解のプロセス}

\begin{itemize}

\item 第一段階:\\命題の検証や推論といった言語実践の中で、有意味/無意味/ナンセンスを弁別する我々の"言語直感"を頼りに分析がなされ、要素命題と論理語が区別される。

\item 第二段階:\\要素命題は名と対象の対に分解される。そして名は、論理形式によってどのような配列が可能であるかチェックされ、対象も論理形式によっていかなる事態の構成要素となるかチェックされた上で分解される。

\end{itemize}

\subsubsection{構成のプロセス}

\begin{itemize}
\item 第一段階:\\ 名の論理形式にしたがって可能な要素命題のすべてが構成される。構成された要素命題は、すべての可能な事態を表現するものとなる。
\item 第二段階:\\ 事態の集合として状況が作られ、可能な状況の全体として論理空間が貼られる。また、命題では論理後によって要素命題から複合命題が作られる。

\end{itemize}

\section{真理操作}


\end{document}
